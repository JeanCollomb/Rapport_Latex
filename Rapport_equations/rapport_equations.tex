\documentclass[a4paper,12pt]{report}

\include{Packages}
\include{Format}


%%%%%%%%%%%%%%%%%%%%%%%%%%%%%%%%%%%%%%%%%%%%%%%%%%%%%%%%%%%%%%%%%%%%%%%%%%%%%%%%%%%%
%%-------------> PAGE DE GARDE INFO
%%%%%%%%%%%%%%%%%%%%%%%%%%%%%%%%%%%%%%%%%%%%%%%%%%%%%%%%%%%%%%%%%%%%%%%%%%%%%%%%%%%%

\author{Auteur}
\newcommand{\validator}{J. COLLOMB}
\title{Rapport d'avancement}
\selectlanguage{french}	
\date{\today}
\newcommand{\thetitleobject}{Comment insérer des équations ?}
\setcounter{tocdepth}{6}
\setcounter{secnumdepth}{6}


%%%%%%%%%%%%%%%%%%%%%%%%%%%%%%%%%%%%%%%%%%%%%%%%%%%%%%%%%%%%%%%%%%%%%%%%%%%%%%%%%%%%
%%-------------> DEBUT DU DOCUMENT 
%%%%%%%%%%%%%%%%%%%%%%%%%%%%%%%%%%%%%%%%%%%%%%%%%%%%%%%%%%%%%%%%%%%%%%%%%%%%%%%%%%%%

\begin{document}

\graphicspath{{Figures/}}

\include{page_de_garde}

%%%%%%%%%%%%%%%%%%%%%%%%%%%%%%%%%%%%%%%%%%%%%%%%%%%%%%%%%%%%%%%%%%%%%%%%%%%%%%%%%%%%
%%-------------> SOMMAIRE
%%%%%%%%%%%%%%%%%%%%%%%%%%%%%%%%%%%%%%%%%%%%%%%%%%%%%%%%%%%%%%%%%%%%%%%%%%%%%%%%%%%%

\renewcommand\contentsname{Sommaire}
\setcounter{chapter}{1}
\tableofcontents
%\listoffigures
%\listoftables


%%%%%%%%%%%%%%%%%%%%%%%%%%%%%%%%%%%%%%%%%%%%%%%%%%%%%%%%%%%%%%%%%%%%%%%%%%%%%%%%%%%%
%%-------------> CORPS DOCUMENT
%%%%%%%%%%%%%%%%%%%%%%%%%%%%%%%%%%%%%%%%%%%%%%%%%%%%%%%%%%%%%%%%%%%%%%%%%%%%%%%%%%%%

%-----------------------------------------------------------------------------------
%-----------------------------------------------------------------------------------
\newpage

\section{Équations en ligne}
Sous \LaTeX, il est possible d'intégrer des équations en ligne, c'est à dire dans le texte. Ces équations ne sont pas numérotés. Pour exemple, voici une équation en ligne : $a \times b~=~a.b$. \\

Pour intégrer une équation en ligne, la commande \LaTeX est : \verb|$ équation $|. \\

Cette commande permet d'intégrer :
\begin{itemize}
\item des équations \verb|$ a \times b = a.b $| $ \longrightarrow a \times b = a.b$;
\item des symboles \verb|$ \Delta $, $ \epsilon $, $ \phi $| $ \longrightarrow  \Delta $, $ \epsilon $, $ \phi $;
\item des indices \verb|$ \Delta T_{maximal} $| $ \longrightarrow  \Delta T_{maximal} $;
\item des exposants \verb|$ \Delta T^{maximal} $| $ \longrightarrow  \Delta T^{maximal} $;
\item des fractions \verb|$ \frac{a}{b} $| $ \longrightarrow  \frac{a}{b} $; 
\item des racines carrées \verb|$ \sqrt{abc} $| $ \longrightarrow  \sqrt{abc} $; 
\item des racines $n^{ième}$ \verb|$ \sqrt[n]{abc} $| $ \longrightarrow  \sqrt[n]{abc} $; 
\item \ldots
\end{itemize}


\section{Équations hors ligne}
Il est également possible d'intégrer des équations hors-ligne, c'est-à-dire séparées du texte. Ces équations sont numérotés. Les équations~\ref{eq_densite_flux_conduction} et \ref{eq_equatio_chaleur} sont des équations hors-ligne. \\

Pour intégrer une équation hors-ligne, la commande \LaTeX est : \\
\verb|\begin{equation}| \\
\verb|\overrightarrow{\varphi_{conductif}} = -\lambda(T) \times \overrightarrow{grad(T)}| \\
\verb|\label{eq_densite_flux_conduction}| \\
\verb|\end{equation}|

\begin{equation}
\overrightarrow{\varphi_{conductif}} = -\lambda(T) \times \overrightarrow{grad(T)}
\label{eq_densite_flux_conduction}
\end{equation}

\verb|\begin{equation}| \\
\verb|\rho (T) \times c_{p} (T) \times \frac{\partial T}{\partial t} = | \\
\verb|\frac{\partial }{\partial x} \times (\lambda (T) \times | \\
\verb|\frac{\partial T}{\partial x}) + \frac{\partial }{\partial y} \times | \\
\verb|(\lambda (T) \times \frac{\partial T}{\partial y}) + | \\
\verb|\frac{\partial }{\partial z} \times (\lambda (T) \times | \\
\verb|\frac{\partial T}{\partial z}) + \dot{E}_{generee}| \\
\verb|\label{eq_equatio_chaleur}| \\
\verb|\end{equation}|

\begin{equation}
\rho (T) \times c_{p} (T) \times \frac{\partial T}{\partial t} = \frac{\partial }{\partial x} \times (\lambda (T) \times \frac{\partial T}{\partial x}) + \frac{\partial }{\partial y} \times (\lambda (T) \times \frac{\partial T}{\partial y}) + \frac{\partial }{\partial z} \times (\lambda (T) \times \frac{\partial T}{\partial z}) + \dot{E}_{generee}
\label{eq_equatio_chaleur}
\end{equation}



\section{Sous équations}
\LaTeX permet également la création de sous-équations. Ces équations sont numérotés. Les équations~\ref{eq_equation_singularite_A1} à \ref{eq_equation_singularite_A3}. \\

Pour intégrer des sous-équations, la commande \LaTeX est : \\
\verb|\begin{subequations}| \\
\verb|\begin{align}| \\
\verb|A &= 0,9 \times  \sin (\delta)  & & \text{ pour } {\delta \leq 70^{\circ}}| \\
\verb|\label{eq_equation_singularite_A1}| \\
\verb|\\| \\
\verb|A &= 1 & & \text{ pour } {\delta = 90^{\circ}}| \\
\verb|\label{eq_equation_singularite_A2}| \\
\verb|\\| \\
\verb|A &= 0,7 + 0,35 \times  \frac{\delta}{90} & & \text{ pour } {\delta \geq 90^{\circ}}| \\
\verb|\label{eq_equation_singularite_A3}| \\
\verb|\end{align}| \\
\verb|\end{subequations} | \\

\begin{subequations}
\begin{align}
A &= 0,9 \times  \sin (\delta)  & & \text{ pour } {\delta \leq 70^{\circ}}
\label{eq_equation_singularite_A1}
\\
A &= 1 & & \text{ pour } {\delta = 90^{\circ}}
\label{eq_equation_singularite_A2}
\\
A &= 0,7 + 0,35 \times  \frac{\delta}{90} & & \text{ pour } {\delta \geq 90^{\circ}}
\label{eq_equation_singularite_A3}
\end{align}
\end{subequations} 



\section{Matrices}
Il est possible sous \LaTeX d'intégrer des matrices. Par exemple, équation~\ref{matrice} et \ref{matrice2}.

\begin{equation}
\begin{bmatrix}
a & b & c \\ 
d & e & f \\ 
g & h & i
\end{bmatrix} 
\label{matrice}
\end{equation}

\begin{equation}
\begin{pmatrix}
1 & 2 \\ 
3 & 4
\end{pmatrix}
\label{matrice2}
\end{equation}



%%%%%%%%%%%%%%%%%%%%%%%%%%%%%%%%%%%%%%%%%%%%%%%%%%%%%%%%%%%%%%%%%%%%%%%%%%%%%%%%%%%%
%%-------------> FIN DU DOCUMENT
%%%%%%%%%%%%%%%%%%%%%%%%%%%%%%%%%%%%%%%%%%%%%%%%%%%%%%%%%%%%%%%%%%%%%%%%%%%%%%%%%%%%

\end{document}