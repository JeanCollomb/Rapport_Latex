\documentclass[a4paper,12pt]{report}

\include{Packages}
\include{Format}


%%%%%%%%%%%%%%%%%%%%%%%%%%%%%%%%%%%%%%%%%%%%%%%%%%%%%%%%%%%%%%%%%%%%%%%%%%%%%%%%%%%%
%%-------------> PAGE DE GARDE INFO
%%%%%%%%%%%%%%%%%%%%%%%%%%%%%%%%%%%%%%%%%%%%%%%%%%%%%%%%%%%%%%%%%%%%%%%%%%%%%%%%%%%%

\author{Auteur}
\newcommand{\validator}{J. COLLOMB}
\title{Rapport d'avancement}
\selectlanguage{french}	
\date{\today}
\newcommand{\thetitleobject}{Comment intégrer des modèles 3D ?}
\setcounter{tocdepth}{6}
\setcounter{secnumdepth}{6}


%%%%%%%%%%%%%%%%%%%%%%%%%%%%%%%%%%%%%%%%%%%%%%%%%%%%%%%%%%%%%%%%%%%%%%%%%%%%%%%%%%%%
%%-------------> DEBUT DU DOCUMENT 
%%%%%%%%%%%%%%%%%%%%%%%%%%%%%%%%%%%%%%%%%%%%%%%%%%%%%%%%%%%%%%%%%%%%%%%%%%%%%%%%%%%%

\begin{document}

\graphicspath{{Figures/}}

\include{page_de_garde}

%%%%%%%%%%%%%%%%%%%%%%%%%%%%%%%%%%%%%%%%%%%%%%%%%%%%%%%%%%%%%%%%%%%%%%%%%%%%%%%%%%%%
%%-------------> SOMMAIRE
%%%%%%%%%%%%%%%%%%%%%%%%%%%%%%%%%%%%%%%%%%%%%%%%%%%%%%%%%%%%%%%%%%%%%%%%%%%%%%%%%%%%

\renewcommand\contentsname{Sommaire}
\setcounter{chapter}{1}
\tableofcontents
%\listoffigures
%\listoftables


%%%%%%%%%%%%%%%%%%%%%%%%%%%%%%%%%%%%%%%%%%%%%%%%%%%%%%%%%%%%%%%%%%%%%%%%%%%%%%%%%%%%
%%-------------> CORPS DOCUMENT
%%%%%%%%%%%%%%%%%%%%%%%%%%%%%%%%%%%%%%%%%%%%%%%%%%%%%%%%%%%%%%%%%%%%%%%%%%%%%%%%%%%%

%-----------------------------------------------------------------------------------
%-----------------------------------------------------------------------------------
\newpage

\section{Générer un modèle 3D}
\subsection{Package et formats requis}
Le package utilisé pour l'intégration d'un modèle 3D est : media9. \\
La commande à placer en début de document est donc : \verb|\usepackage{media9}| \\

Le package media9 permet l'intégration des fichiers \verb|.u3d| et \verb|.prc|. \\
Il est donc nécessaire de générer un modèle 3D à ce format. \\

Des logiciels ne permettent l'export à ce format (ex : SolidWorks), il est donc nécessaire de mettre en place des étapes intermédiaires. Une proposition de démarche est présentée ci-après.

\subsection{Suggestion de démarche}

La démarche proposée repose sur 4 étapes :
\begin{enumerate}
\item Générer le modèle 3D;
\item Exporter dans un format d'échange : \verb|.stl| par exemple;
\item Convertir ce fichier au format \verb|.u3d| à l'aide d'un logiciel externe, \href{http://www.meshlab.net/}{Meshlab} par exemple;
\item Intégrer ce fichier dans \LaTeX.
\end{enumerate}

\subsection{Code \LaTeX}
Un exemple de code est présenté ci-après. Pour plus de détails, voici la \href{http://texdoc.net/texmf-dist/doc/latex/media9/media9.pdf}{documentation} du package avec des exemples. \\

\verb|% Intégration du modèle| \\
\verb|\begin{figure}[hbtp] | \\
\verb|\begin{center} | \\
\verb|\includemedia[ | \\
\verb|label=figure_modele, | \\
\verb|width=0.85\linewidth,height=0.85\linewidth, | \\
\verb|activate=pagevisible, | \\
\verb|3Dpartsattrs=keep, | \\
\verb|3Dnavpane, | \\
\verb|3Dviews={Figures/dynamique/modele.vues} | \\
\verb|]{\includegraphics{modele.pdf}}{/dynamique/modele.u3d} | \\

\verb|% Création des boutons cliquables| \\
\verb|\mediabutton[3Dgotoview=figure_modele:0]{\fbox{Vue par défaut}} | \\
\verb|\mediabutton[3Dgotoview=figure_modele:1]{\fbox{Vue de face}} | \\
\verb|\mediabutton[3Dgotoview=figure_modele:2]{\fbox{Vue du maillage}} | \\
\verb|\mediabutton[3Dgotoview=figure_modele:3]{\fbox{Vue en transparence}} | \\

\verb|\end{center}   | \\
\verb|\caption{Exemple de vues dynamiques | \\
\verb|\label{figure_modele_dynamique}}  | \\
\verb|\end{figure} | \\


\section{Exemple}
Un exemple de vue dynamique / interactive est présenté Figure.~\ref{figure_modele_dynamique}. Pour son bon fonctionnement, il est nécessaire d'activer les formulaires et cliquez sur l'image si nécessaire pour charger la visualisation; les zooms et translations sur le modèle sont disponibles.

\begin{figure}[hbtp]
\begin{center}
\includemedia[
label=figure_modele,
width=0.85\linewidth,height=0.85\linewidth,
activate=pagevisible,
3Dpartsattrs=keep,
3Dnavpane,
3Dviews={Figures/dynamique/modele.vues}
]{\includegraphics{modele.pdf}}{/dynamique/modele.u3d}

\mediabutton[3Dgotoview=figure_modele:0]{\fbox{Vue par défaut}}
\mediabutton[3Dgotoview=figure_modele:1]{\fbox{Vue de face}}
\mediabutton[3Dgotoview=figure_modele:2]{\fbox{Vue du maillage}}
\mediabutton[3Dgotoview=figure_modele:3]{\fbox{Vue en transparence}}

\end{center}  
\caption{Exemple de vues dynamiques
\label{figure_modele_dynamique}} 
\end{figure}



%%%%%%%%%%%%%%%%%%%%%%%%%%%%%%%%%%%%%%%%%%%%%%%%%%%%%%%%%%%%%%%%%%%%%%%%%%%%%%%%%%%%
%%-------------> FIN DU DOCUMENT
%%%%%%%%%%%%%%%%%%%%%%%%%%%%%%%%%%%%%%%%%%%%%%%%%%%%%%%%%%%%%%%%%%%%%%%%%%%%%%%%%%%%

\end{document}