\documentclass[a4paper,12pt]{report}

\include{Packages}
\include{Format}


%%%%%%%%%%%%%%%%%%%%%%%%%%%%%%%%%%%%%%%%%%%%%%%%%%%%%%%%%%%%%%%%%%%%%%%%%%%%%%%%%%%%
%%-------------> PAGE DE GARDE INFO
%%%%%%%%%%%%%%%%%%%%%%%%%%%%%%%%%%%%%%%%%%%%%%%%%%%%%%%%%%%%%%%%%%%%%%%%%%%%%%%%%%%%

\author{Auteur}
\newcommand{\validator}{J. COLLOMB}
\title{Rapport d'avancement}
\selectlanguage{french}	
\date{\today}
\newcommand{\thetitleobject}{Comment intégrer une animation ?}
\setcounter{tocdepth}{6}
\setcounter{secnumdepth}{6}


%%%%%%%%%%%%%%%%%%%%%%%%%%%%%%%%%%%%%%%%%%%%%%%%%%%%%%%%%%%%%%%%%%%%%%%%%%%%%%%%%%%%
%%-------------> DEBUT DU DOCUMENT 
%%%%%%%%%%%%%%%%%%%%%%%%%%%%%%%%%%%%%%%%%%%%%%%%%%%%%%%%%%%%%%%%%%%%%%%%%%%%%%%%%%%%

\begin{document}

\graphicspath{{Figures/}}

\include{page_de_garde}

%%%%%%%%%%%%%%%%%%%%%%%%%%%%%%%%%%%%%%%%%%%%%%%%%%%%%%%%%%%%%%%%%%%%%%%%%%%%%%%%%%%%
%%-------------> SOMMAIRE
%%%%%%%%%%%%%%%%%%%%%%%%%%%%%%%%%%%%%%%%%%%%%%%%%%%%%%%%%%%%%%%%%%%%%%%%%%%%%%%%%%%%

\renewcommand\contentsname{Sommaire}
\setcounter{chapter}{1}
\tableofcontents
%\listoffigures
%\listoftables


%%%%%%%%%%%%%%%%%%%%%%%%%%%%%%%%%%%%%%%%%%%%%%%%%%%%%%%%%%%%%%%%%%%%%%%%%%%%%%%%%%%%
%%-------------> CORPS DOCUMENT
%%%%%%%%%%%%%%%%%%%%%%%%%%%%%%%%%%%%%%%%%%%%%%%%%%%%%%%%%%%%%%%%%%%%%%%%%%%%%%%%%%%%

%-----------------------------------------------------------------------------------
%-----------------------------------------------------------------------------------
\newpage

\section{Préparation amont}
\subsection{Principe}
Tout comme les films, une animation est générée par l'assemblage de plusieurs images à une cadence plus ou moins importante. Une animation fluide possède une fréquence de l'ordre de 25 images/seconde, mais pour certaines applications, des fréquences inférieures ou supérieures peuvent être requises.

\subsection{Génération des images}
Les images peuvent être générées de différentes manières :
\begin{itemize}
\item sous Python/Matlab\ldots;
\item sous logiciel éléments finis;
\item \ldots
\end{itemize}

Pour un bon fonctionnement sous \LaTeX, il est nécessaire de nommer les fichiers sous un format du type : \verb|nom_du_fichier_1.pdf| ; \verb|nom_du_fichier_2.pdf| ; \ldots

\begin{figure}[hbtp]
	\centering
	\begin{subfigure}[b]{0.25\textwidth}
		\centering
		\includegraphics[scale=0.12]{Figures/animation/Figure_1.pdf}
	\end{subfigure}
	\qquad
	\begin{subfigure}[b]{0.25\textwidth}
		\centering
		\includegraphics[scale=0.12]{Figures/animation/Figure_3.pdf}
	\end{subfigure}
	\qquad
	\begin{subfigure}[b]{0.25\textwidth}
		\centering
		\includegraphics[scale=0.12]{Figures/animation/Figure_5.pdf}
	\end{subfigure}
	\qquad
	\begin{subfigure}[b]{0.25\textwidth}
		\centering
		\includegraphics[scale=0.12]{Figures/animation/Figure_7.pdf}
	\end{subfigure}
	\qquad
	\begin{subfigure}[b]{0.25\textwidth}
		\centering
		\includegraphics[scale=0.12]{Figures/animation/Figure_9.pdf}
	\end{subfigure}
	\qquad
	\begin{subfigure}[b]{0.25\textwidth}
		\centering
		\includegraphics[scale=0.12]{Figures/animation/Figure_11.pdf}
	\end{subfigure}
	\qquad
	\begin{subfigure}[b]{0.25\textwidth}
		\centering
		\includegraphics[scale=0.12]{Figures/animation/Figure_13.pdf}
	\end{subfigure}
	\qquad
	\begin{subfigure}[b]{0.25\textwidth}
		\centering
		\includegraphics[scale=0.12]{Figures/animation/Figure_15.pdf}
	\end{subfigure}
	\caption{Exemple de figure générées} 
	\label{figure_deux_logo_SYMME}
\end{figure}


\FloatBarrier
%###########################################################################
%###########################################################################

\section{Code \LaTeX}
\subsection{Préparation \LaTeX}
La génération d'animation est possible grâce à l'utilisation du package "animate". Il faut donc ajouter en début de document : \verb|\usepackage{animate}|. Des informations complémentaires sont accessibles sur la \href{http://tug.ctan.org/macros/latex/contrib/animate/animate.pdf}{page de documentation} du package.\\

Ci-dessous, un exemple de code \LaTeX pour la création d'une animation. \\


\verb|\begin{figure}[hbtp]|  \\
\verb|\begin{center}|  \\
\verb|\animategraphics[|  \\
\verb|autoplay, 		% Lecture automatique à l'affichage de la page |  	\\
\verb|loop,				% Boucle de lecture de l'animation |  		\\
\verb|poster=first, |  \\
\verb|height=13cm, 		% Dimension de l'animation |  \\
\verb|width=18cm ,		% Dimension de l'animation |  \\
\verb|controls]{5}{/animation/Figure_}{1}{25} % Animation de l'image 1 à 25 avec 5i/s |  \\
\verb|\caption{Exemple d'animation (Adobe reader 10 minimum requis)} |  \\
\verb|\label{Fig_animation} |  \\
\verb|\end{center} |   \\
\verb|\end{figure}|  \\


\subsection{Avertissement}
Pour le bon fonctionnement de l'animation sur le document pdf final, il est nécessaire de posséder Adobe reader X au minimum. \\

L'impression papier quant à elle est censé faire figuer l'image déterminée par la commande "poster".

\subsection{Exemple d'animation}

Un exemple d'animation est présenté Figure.~\ref{Fig_animation}. \\

Cette animation représente la chauffe d'une structure 2D après l'exécution d'un calcul éléments finis sur le logiciel Ansys. \\


\begin{figure}[hbtp]
\begin{center}
\animategraphics[
autoplay, 
loop,
poster=first,
height=13cm, 
width=18cm ,
controls]{5}{/animation/Figure_}{1}{25}
\caption{Exemple d'animation (Adobe reader 10 minimum requis)} 
\label{Fig_animation} 
\end{center}  
\end{figure}


%%%%%%%%%%%%%%%%%%%%%%%%%%%%%%%%%%%%%%%%%%%%%%%%%%%%%%%%%%%%%%%%%%%%%%%%%%%%%%%%%%%%
%%-------------> FIN DU DOCUMENT
%%%%%%%%%%%%%%%%%%%%%%%%%%%%%%%%%%%%%%%%%%%%%%%%%%%%%%%%%%%%%%%%%%%%%%%%%%%%%%%%%%%%

\end{document}