\documentclass[a4paper,12pt]{report}

\include{Packages}
\include{Format}


%%%%%%%%%%%%%%%%%%%%%%%%%%%%%%%%%%%%%%%%%%%%%%%%%%%%%%%%%%%%%%%%%%%%%%%%%%%%%%%%%%%%
%%-------------> PAGE DE GARDE INFO
%%%%%%%%%%%%%%%%%%%%%%%%%%%%%%%%%%%%%%%%%%%%%%%%%%%%%%%%%%%%%%%%%%%%%%%%%%%%%%%%%%%%

\author{Auteur}
\newcommand{\validator}{J. COLLOMB}
\title{Rapport d'avancement}
\selectlanguage{french}	
\date{\today}
\newcommand{\thetitleobject}{Petits guide pratique pour les doctorants}
\setcounter{tocdepth}{6}
\setcounter{secnumdepth}{6}


%%%%%%%%%%%%%%%%%%%%%%%%%%%%%%%%%%%%%%%%%%%%%%%%%%%%%%%%%%%%%%%%%%%%%%%%%%%%%%%%%%%%
%%-------------> DEBUT DU DOCUMENT 
%%%%%%%%%%%%%%%%%%%%%%%%%%%%%%%%%%%%%%%%%%%%%%%%%%%%%%%%%%%%%%%%%%%%%%%%%%%%%%%%%%%%

\begin{document}

\graphicspath{{Figures/}}

\include{page_de_garde}

%%%%%%%%%%%%%%%%%%%%%%%%%%%%%%%%%%%%%%%%%%%%%%%%%%%%%%%%%%%%%%%%%%%%%%%%%%%%%%%%%%%%
%%-------------> SOMMAIRE
%%%%%%%%%%%%%%%%%%%%%%%%%%%%%%%%%%%%%%%%%%%%%%%%%%%%%%%%%%%%%%%%%%%%%%%%%%%%%%%%%%%%

\renewcommand\contentsname{Sommaire}
\setcounter{chapter}{1}
\tableofcontents
%\listoffigures
%\listoftables


%%%%%%%%%%%%%%%%%%%%%%%%%%%%%%%%%%%%%%%%%%%%%%%%%%%%%%%%%%%%%%%%%%%%%%%%%%%%%%%%%%%%
%%-------------> CORPS DOCUMENT
%%%%%%%%%%%%%%%%%%%%%%%%%%%%%%%%%%%%%%%%%%%%%%%%%%%%%%%%%%%%%%%%%%%%%%%%%%%%%%%%%%%%

%-----------------------------------------------------------------------------------
%-----------------------------------------------------------------------------------
\newpage
\section{Personnes clés d'une thèse}



%-----------------------------------------------------------------------------------
%-----------------------------------------------------------------------------------
\FloatBarrier
\newpage
\section{Moments clés d'une thèse}
\subsection{La recherche bibliographique}

\textit{Le travail de recherche et l'écriture d'un texte scientifique (rapport, article, thèse,…) exigent une recherche d'informations approfondie qui prend directement appui sur les travaux antérieurs. L'information choisie et exploitée permet de développer une réflexion personnelle et chaque document, retenu et analysé, contribue à la crédibilité scientifique du travail présenté. \\}

\textit{Afin de faciliter la réflexion et le travail de recherche des lecteurs, qui à leur tour vont vouloir croiser leurs informations, il est indispensable de référencer correctement les travaux cités dans le texte en rédigeant une partie intitulée « Bibliographie » ou « Références bibliographiques ». \\}

\textit{La bibliographie d'un document permet de connaître : (i)les travaux qui ont été utilisés pour le travail de recherche et la rédaction ; (ii) l’état de la littérature sur un sujet pendant une période déterminée ;(iii) les auteurs, titres de revue, sites web… spécialisés dans un domaine.} \\

Source : \href{http://www.ajar-online.fr/thesememoire-2-recherche-bibliographique/}{AJAR Paris}    \\

Quelques endroits pour effectuer sa recherche bibliographique :
\begin{itemize}
\item \href{https://scholar.google.fr/}{Google Scholar};
\item \href{https://www-sciencedirect-com.camphrier-1.grenet.fr/}{Science Direct};
\item \href{https://link-springer-com.camphrier-1.grenet.fr/}{Springer};
\item \href{https://www-techniques-ingenieur-fr.camphrier-1.grenet.fr/}{Techniques de l'ingénieur}
\item \href{https://hal.archives-ouvertes.fr/}{HAL}.
\end{itemize}

\ \\

\begin{figure}[hbtp]
	\centering
	\def\svgwidth{1\columnwidth}
	\fontsize{10pt}{10pt}\selectfont\input{Figures/image_bibliographie.pdf_tex}
	\caption{Acquisition de connaissances}
	\label{figure_bouquins}
\end{figure}



%--------------------------------------------------
\FloatBarrier
\subsection{Les travaux de recherche}

L’essentiel de l’activité doctorale consiste en un travail de recherche novateur limité dans le temps (3 ans), encadré par un directeur de recherches doctorales, au sein d’une unité de recherche. Elle se conclut par la rédaction d’une thèse et par sa soutenance, c’est-à-dire une restitution synthétique des travaux scientifiques  effectués, validés par la communauté scientifique. \\

Le travail de recherche doit être novateur, c’est-à-dire nouveau et entraînant une révision, une transformation de l’existant. Il est structuré par l’invention ou la construction de savoir-faire, de technologies, ou au sens large, d’outils – qu’ils soient conceptuels ou méthodologiques – innovants, et dont la conception a permis l’élaboration de nouvelles connaissances ou l’extension des capacités d’action. \\

\textit{Quelques actions liées au travail de thèse :
\begin{itemize}
\item Réfléchir, développer une analyse critique et structurer ses pensées;
\item Conduire des recherches / enquêtes (collecter de l'information, la traiter, puis la restituer de manière cohérente);
\item Écrire de manière correcte, compréhensible et articulée;
\item Présenter son travail;
\item Respecter les délais;
\item Travailler à la fois de manière autonome et en équipe;
\item Gérer un projet (la thèse) du début à la fin;
\item Prendre des initiatives;
\item Développer un « réseau » et contribuer à son animation;
\item Faire preuve de détermination et d'endurance.
\end{itemize}}

\ \\

Source : \href{https://act.hypotheses.org/504}{Les aspects concrets de la thèse}

%--------------------------------------------------
\FloatBarrier
\subsection{Soutenance et rédaction finale}

\textit{La soutenance de thèse est l'épreuve universitaire concluant la thèse de doctorat. Il s'agit en général d'un examen oral prenant la forme d'une présentation effectuée par le candidat au titre de docteur durant laquelle il expose ses travaux de recherche devant un jury de spécialistes.} \\

\textit{Le doctorant envoie en principe son mémoire de thèse aux membres du jury plusieurs semaines avant la soutenance, afin que ceux-ci aient le temps de le consulter, et le dépose officiellement auprès de son université trois semaines avant la soutenance1, en indiquant une date prévisionnelle de soutenance.} \\

Source : \href{https://fr.wikipedia.org/wiki/Soutenance_de_th\%C3\%A8se}{Wikipedia}


%--------------------------------------------------
\FloatBarrier
\subsection{Outils pour la bibliographie}

L'utilisation d'un logiciel de gestion bibliographique est plus que recommandé dans le cadre de la thèse. Ces outils permettent généralement d'organiser sa bibliographie (en dossier par exemple), d'obtenir l'ensemble des informations nécessaires pour les citations dans les rapports scientifiques, d'exporter et d'intégrer aisément la bibliographie à un rapport\ldots Il existe différents logiciels, comme par exemple : \href{https://www.mendeley.com/}{Mendeley}, \href{https://www.zotero.org/}{Zotero},\ldots Un exemple de l'interface de Mendeley est visible Figure~\ref{figure_mendeley}.


\begin{figure}[hbtp]
	\centering
	\def\svgwidth{1\columnwidth}
	\fontsize{10pt}{10pt}\selectfont\input{Figures/mendeley.pdf_tex}
	\caption{Interface de Mendeley}
	\label{figure_mendeley}
\end{figure}


%--------------------------------------------------
\FloatBarrier
\subsection{Outils pour le tracé de figures}

Il est courant pour les doctorants de devoir exploiter des résultats issus d'expérimentations (relevés de température, mesures de déplacement, corrélation d'images\ldots). \href{https://www.anaconda.com/download/}{Python} est un logiciel libre avec lequel il est possible d'exploiter divers résultats et de les afficher sous forme de Figure (\href{https://github.com/JeanCollomb/Python_plot}{Quelques exemples simples}). \\

La Figure~\ref{figure_exemples_python} présente quelques exemples de Figures pouvant être générées à l'aide de Python : (i) courbe 3D Figure~\ref{figure_courbe_3D}; (ii) subplot Figure~\ref{figure_subplot}; (iii) diagramme polaire Figure~\ref{figure_polaire}.

\begin{figure}[hbtp]
	\centering
	\begin{subfigure}[b]{0.5\textwidth}
		\centering
		\def\svgwidth{\columnwidth}
		\fontsize{10pt}{10pt}\selectfont\input{Figures/Figure_1.pdf_tex}
		\caption{Exemple Courbe 3D} 
		\label{figure_courbe_3D}
	\end{subfigure}
	\qquad
	\begin{subfigure}[b]{0.5\textwidth}
		\centering
		\def\svgwidth{\columnwidth}
		\fontsize{10pt}{10pt}\selectfont\input{Figures/Figure_2.pdf_tex}
		\caption{Exemple subplot} 
		\label{figure_subplot}
	\end{subfigure}
	\qquad
	\begin{subfigure}[b]{0.5\textwidth}
		\centering
		\def\svgwidth{\columnwidth}
		\fontsize{10pt}{10pt}\selectfont\input{Figures/Figure_3.pdf_tex}
		\caption{Exemple diagramme polaire} 
		\label{figure_polaire}
	\end{subfigure}
	\caption{Exemple de Figures générées sous Python} 
	\label{figure_exemples_python}
\end{figure}



%-----------------------------------------------------------------------------------
%-----------------------------------------------------------------------------------
\FloatBarrier
\newpage
\section{L'organisation, une nécessité}
\subsection{Un projet sur 3 ans}



%--------------------------------------------------
\FloatBarrier
\subsection{De multiples acteurs}



%--------------------------------------------------
\FloatBarrier
\subsection{De nombreux livrables}




%--------------------------------------------------
\FloatBarrier
\subsection{Outils pour s'organiser}






%-----------------------------------------------------------------------------------
%-----------------------------------------------------------------------------------
\FloatBarrier
\newpage
\section{La rédaction... un passage obligé}
\subsection{Point d'avancement}

\textit{Les rapports de progression sont très importants pour gérer un projet professionnel ou universitaire. De plus, ils vous serviront à informer plus facilement vos supérieurs, vos collègues ou vos clients sur la progression du projet que vous réalisez. Votre rapport devra préciser le travail accompli et les étapes qui restent à franchir pour mener le projet à terme.} \\

Source : \href{https://fr.wikihow.com/r\%C3\%A9diger-un-rapport-d\%27avancement}{WikiHow} \\

\underline{\textbf{Exemple de rapport d'avancement :}}\\
\begin{itemize}
\item Après une phase de bibliographie;
\item Après une étude de risques;
\item Après une une phase de développement;
\item Après la réalisation d'une étude;
\item Après la visite des locaux d'un partenaire;
\item Auprès d'un organisme finançant le travail;
\item \ldots
\end{itemize}



%--------------------------------------------------
\FloatBarrier
\subsection{Rédaction scientifique}

\textit{L'expression « publication scientifique » regroupe plusieurs types de communications scientifiques et/ou techniques avancées que les chercheurs scientifiques font de leurs travaux en direction de leur pairs et d'un public de spécialistes. Ces publications ayant subi une forme d'examen de la rigueur de la méthode scientifique employée pour ces travaux, comme l'examen par un comité de lecture indépendant constitué de pairs. } \\

Source : \href{https://fr.wikipedia.org/wiki/Publication_scientifique}{Wikipedia} \\

\underline{\textbf{Exemple de rédactions scientifiques :}}\\
\begin{itemize}
\item Rapport final de thèse;
\item Articles de revues scientifiques à comité de lecture;
\item Articles de congrès à comité de lecture;
\item \ldots
\end{itemize}

\ \\

Bien entendu, \LaTeX rend possible la citation de références uniques comme ici \cite{Collomb2018a} et là \cite{Collomb2017}, ou multiples \cite{Collomb2018,Collomb2017a}.



%--------------------------------------------------
\FloatBarrier
\subsection{\LaTeX, une alternative à Word}

\subsubsection{Qu'est ce que \LaTeX ?}
\LaTeX n'est pas un traitement de texte WYSIWYG (What You See Is What You Get), mais un langage de programmation. Initialement développé pour des applications mathématiques, il permet aujourd'hui la création de rapport de manière aisée. Il est ainsi possible de réaliser : CV, lettres, rapports, livres, thèses, publications\ldots L'auteur se concentre sur le contenu, \LaTeX~sur le rendu. \\
Tout comme Word et LibreOffice, \LaTeX~dispose d'une communauté importante, ce qui permet d'obtenir des réponses aux questions les plus courantes.

\subsubsection{Avantages et Inconvénients}

Le Tableau~\ref{table_avantages_inconvenient_latex} présente les principaux avantages et inconvénients liés à \LaTeX.

\begin{table}[hbtp]
\resizebox{\textwidth}{!}{%
\begin{tabular}{|c|c|}
\hline
\textbf{Avantages}                      & \textbf{Inconvénients}                                              \\ \hline
Stabilité                               & Création de tableaux                                                \\ \hline
Gestion des références                  & Positionnement des objets (tables, figures)                         \\ \hline
Gestion de la mise en forme automatique & Nécessité d'apprentissage initial                                   \\ \hline
Qualité du document final               & Ajout de commentaires/modifications par un relecteur plus difficile \\ \hline
Légèreté du document                    & Création du modèle du document                                      \\ \hline
...                                     &                                                                     \\ \hline
\end{tabular}%
}
\caption{Principaux avantages et inconvénients de \LaTeX}
\label{table_avantages_inconvenient_latex}
\end{table}

Il est important de noter que la plupart des inconvénients listés Tableau~\ref{table_avantages_inconvenient_latex} peuvent être surmonté de manière simple. Le Tableau~\ref{table_astuces_remedes} présente quelques astuces à ces problématiques.

\begin{table}[hbtp]
\resizebox{\textwidth}{!}{%
\begin{tabular}{|c|c|}
\hline
\textbf{Inconvénients}                                              & \textbf{Astuces / Remèdes}                                                                                                          \\ \hline
Création de tableaux                                                & \href{http://www.tablesgenerator.com/latex_tables}{Générateur de tableau} \\ \hline
Positionnement des objets (tables, figures)                         & \begin{tabular}[c]{@{}c@{}}Commande \textbackslash{}\textbackslash{}FloatBarrier\\ Option de positionnement {[}hbtp{]}\end{tabular} \\ \hline
Nécessité d'apprentissage initial                                   & \begin{tabular}[c]{@{}c@{}}Temps d'apprentissage limité\\ Gain de temps certain par la suite\end{tabular}                           \\ \hline
Ajout de commentaires/modifications par un relecteur plus difficile & Possibilité de commenter le pdf généré                                                                                              \\ \hline
Création du modèle du document                                      & Modèles disponibles en ligne                                                                                                        \\ \hline
\end{tabular}%
}
\caption{Quelques astuces et remèdes}
\label{table_astuces_remedes}
\end{table}


\FloatBarrier
\subsubsection{Comment rédiger sous \LaTeX}

Il faut tout d'abord installer :
\begin{enumerate}
\item une distribution TeX, par exemple \href{https://miktex.org/}{MiKTeX};
\item un éditeur de texte, par exemple \href{http://www.xm1math.net/texmaker/index_fr.html}{TeXMaker}\footnote{D'autres éditeurs existent, à vous de trouvez celui qui vous convient.} dont l'interface est amicale pour l'utilisateur (Figure~\ref{figure_tekmaker}).
\end{enumerate}

Il est également possible de réaliser sa rédaction \LaTeX sur internet (donc sans installation), sur des plateformes du type : \href{https://www.overleaf.com/}{Overleaf}. \\

Pour faciliter l'apprentissage, il peut être judicieux d'étudier des codes \LaTeX~disponibles en ligne par exemple et d'observer le fonctionnement global, ainsi que le rôle des différentes fonctions.\\

\underline{\textbf{Suggestion :}} \\
Pour les Figures :
\begin{enumerate}
\item Préférer la génération d'images vectorielles (\verb|.eps|, \verb|.svg|);
\item Convertir les images en fichiers \verb|.pdf_tex| à l'aide de \href{https://inkscape.org/fr/}{Inkscape};
\item Intégrer uniquement des images \verb|.pdf_tex| dans le document \LaTeX (minimiser les formats différents et le sources d'erreurs).
\end{enumerate}

\ \\

\underline{\textbf{Astuce :}} \\
Il est possible de réaliser des illustrations sur Powerpoint (et de bénéficier de sa facilité d'utilisation), d'enregistrer l'image en \verb|.emf|, puis d'importer ce fichier dans Inkscape pour la génération du \verb|pdf_tex|.

\begin{figure}[hbtp]
	\centering
	\def\svgwidth{1\columnwidth}
	\fontsize{10pt}{10pt}\selectfont\input{Figures/texmaker.pdf_tex}
	\caption{Interface TexMaker}
	\label{figure_tekmaker}
\end{figure}



\FloatBarrier
\subsubsection{Quelques exemples de fonctions sous \LaTeX}

Il est possible d'intégrer des Figures multiples, comme visible Figure~\ref{figure_deux_logo_SYMME}. Bien entendu, un renvoi peut être réalisé sur la sous-figure~\ref{figure_ancien_logo_SYMME} ou sur la sous-figure~\ref{figure_nouveau_logo_SYMME}. \\

\begin{figure}[hbtp]
	\centering
	\begin{subfigure}[b]{0.3\textwidth}
		\centering
		\def\svgwidth{\columnwidth}
		\fontsize{10pt}{10pt}\selectfont\input{Figures/logo_symme_old.pdf_tex}
		\caption{Ancien logo du SYMME} 
		\label{figure_ancien_logo_SYMME}
	\end{subfigure}
	\qquad
	\begin{subfigure}[b]{0.15\textwidth}
		\centering
		\def\svgwidth{\columnwidth}
		\fontsize{10pt}{10pt}\selectfont\input{Figures/Logo-Symme.pdf_tex}
		\caption{Nouveau logo du SYMME} 
		\label{figure_nouveau_logo_SYMME}
	\end{subfigure}
	\caption{Historique des logos SYMME} 
	\label{figure_deux_logo_SYMME}
\end{figure}


\newpage
\FloatBarrier
Un exemple d'animation est présenté Figure.~\ref{Fig_animation}. Adobe reader X minimum requis.

\begin{figure}[hbtp]
\begin{center}
\animategraphics[autoplay, loop,
poster=first,
height=13cm, 
width=18cm ,
controls]{5}{/animation/Figure_}{1}{25}
\caption{Exemple d'animation (Adobe reader 10 minimum requis)} 
\label{Fig_animation} 
\end{center}  
\end{figure}


\newpage
\FloatBarrier
Un exemple de vue dynamique / intéractive est présenté Figure.~\ref{figure_modele_dynamique}. Pour son bon fonctionnement, il est nécessaire d'activer les formulaires et cliquez sur l'image si nécessaire pour charger la visualisation; les zooms et translations sur le modèle sont disponibles.

% ETAPE 1
%\begin{center}
%\includemedia[
%width=0.75\linewidth,height=0.75\linewidth,
%activate=pageopen,
%3Dmenu,
%3Dtoolbar,3Dpartsattrs=keep,3Dnavpane
%]{}{dynamique/modele.u3d}
%\end{center}

%ETAPE 2
\begin{figure}[hbtp]
\begin{center}
\includemedia[
label=figure_modele,
width=0.85\linewidth,height=0.85\linewidth,
activate=pagevisible,
3Dpartsattrs=keep,
3Dnavpane,
3Dviews={Figures/dynamique/modele.vues}
]{\includegraphics{modele.pdf}}{/dynamique/modele.u3d}

\mediabutton[3Dgotoview=figure_modele:0]{\fbox{Vue par défaut}}
\mediabutton[3Dgotoview=figure_modele:1]{\fbox{Vue de face}}
\mediabutton[3Dgotoview=figure_modele:2]{\fbox{Vue du maillage}}
\mediabutton[3Dgotoview=figure_modele:3]{\fbox{Vue en transparence}}

\end{center}  
\caption{Exemple de vues dynamiques
\label{figure_modele_dynamique}} 
\end{figure}


\newpage
\FloatBarrier
\LaTeX rend possible l'intégration d'images vectorielles, comme visible Figure~\ref{figure_vectorielle}. A la différence des "images matricielles" constituées de pixels, les images vectorielles sont des images numériques dans lesquelles il est possible de zoomer sans perte de qualité (sans apparition de pixel). Faites l'essai sur la Figure~\ref{figure_vectorielle} !

\begin{figure}[hbtp]
	\centering
	\def\svgwidth{1\columnwidth}
	\fontsize{10pt}{10pt}\selectfont\input{Figures/image_vectorielle.pdf_tex}
	\caption{Exemple d'image vectorielle}
	\label{figure_vectorielle}
\end{figure}



%-----------------------------------------------------------------------------------
%-----------------------------------------------------------------------------------
\FloatBarrier
\newpage

\bibliographystyle{plain}
\bibliography{bibliographie}


%%%%%%%%%%%%%%%%%%%%%%%%%%%%%%%%%%%%%%%%%%%%%%%%%%%%%%%%%%%%%%%%%%%%%%%%%%%%%%%%%%%%
%%-------------> FIN DU DOCUMENT
%%%%%%%%%%%%%%%%%%%%%%%%%%%%%%%%%%%%%%%%%%%%%%%%%%%%%%%%%%%%%%%%%%%%%%%%%%%%%%%%%%%%

\end{document}